\hypertarget{clase-3-sql-nivel-medio-1}{%
\section{Clase 3: SQL nivel medio (1)}\label{clase-3-sql-nivel-medio-1}}

\begin{frame}[fragile]{Joins}
\protect\hypertarget{joins}{}
Los joins se utilizan para combinar dos o más tablas en una sola
consulta. Los tipos de joins más comunes son INNER JOIN, LEFT JOIN,
RIGHT JOIN y FULL OUTER JOIN.

Ejemplo:

\begin{verbatim}
SELECT column1, column2, column3 
FROM table1 
INNER JOIN table2 
ON table1.column1 = table2.column1;
\end{verbatim}

Esta consulta combina los datos de table1 y table2 en una sola consulta
utilizando un INNER JOIN basado en la columna column1.
\end{frame}

\begin{frame}[fragile]{Operadores de conjuntos: UNION, INTERSECT,
EXCEPT}
\protect\hypertarget{operadores-de-conjuntos-union-intersect-except}{}
Los operadores de conjuntos se utilizan para combinar resultados de dos
o más consultas en una sola consulta.

Ejemplo:

\begin{verbatim}
SELECT column1 
FROM table1 
UNION 
SELECT column1 
FROM table2;
\end{verbatim}

Esta consulta combina los valores de column1 de table1 y table2 en una
sola columna utilizando el operador de conjunto UNION.
\end{frame}
