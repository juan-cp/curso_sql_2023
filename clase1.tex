\hypertarget{clase-1-introducciuxf3n-a-sql}{%
\section{Clase 1: Introducción a
SQL}\label{clase-1-introducciuxf3n-a-sql}}

\begin{frame}{¿Qué es SQL?}
\protect\hypertarget{quuxe9-es-sql}{}
SQL significa Structured Query Language (Lenguaje de Consulta
Estructurado) y es un lenguaje de programación utilizado para
administrar bases de datos relacionales. SQL te permite realizar
diversas operaciones en los datos almacenados en las bases de datos,
como consultas, inserciones, actualizaciones y eliminaciones de datos.
\end{frame}

\begin{frame}{Bases de datos relacionales y tablas}
\protect\hypertarget{bases-de-datos-relacionales-y-tablas}{}
Una base de datos relacional es un tipo de base de datos que almacena
los datos en tablas. Una tabla está compuesta por filas y columnas,
donde cada columna representa un tipo de datos diferente, y cada fila
representa un solo registro.

Una BBDD puede contener varias tablas, y las tablas estar relacionadas
entre sí por campos que comparten, llaves (un tipo de campo que
revisaremos más adelante) y relaciones de cardinalidad entre sus
registros (que también veremos más adelante). El conjunto de tablas y
relaciones en una BBDD se conoce como el \textbf{Esquema de una BBDD} y
se puede representar gráficamente en un diagrama que se conoce como
\textbf{diagrama Entidad-Relación}

Ejemplo de diagrama ER:
https://docs.staruml.io/working-with-additional-diagrams/entity-relationship-diagram
\end{frame}

\begin{frame}[fragile]{Tipos de Datos}
\protect\hypertarget{tipos-de-datos}{}
Las bases de datos en las que se puede consultar con SQL se les conoce
como bases de datos relacionales, mientras que los datos que se
almacenan en ellas son datos estructurados; es decir, corresponden a un
tipo bien específico de dato que caerá en las siguientes categorías:

Tipos Numéricos - \texttt{INT}: Número enteros, en un rango de
\([-2^{31},2^{31}]\) - \texttt{BIGINT}: Número entero entre
\([-2^{64},2^{64}]\) - \texttt{BIT}: Para un número entero que puede ser
0 ó 1. - \texttt{FLOAT}: Números decimales

Tipos de Texto/Fecha - \texttt{CHAR}: Este tipo es para cadenas de
longitud fija. Su longitud va desde 1 a 255 caracteres. Siempre se
ocupara el largo que hayamos dado (añadiendo espacios en el caso que
faltasen caracteres). - \texttt{VARCHAR}: Para una cadena de caracteres
de longitud variable de hasta 8.060. - \texttt{NVARCHAR}: texto de
longitud variable que puede tener hasta 65.535 caracteres. -
\texttt{DATE}: Para almacenar fechas. El formato por defecto es
yyyymmdd. - \texttt{DATETIME}: Combinacion de fecha y hora.
\end{frame}

\begin{frame}[fragile]{Data Definition Language}
\protect\hypertarget{data-definition-language}{}
Las declaraciones DDL o Data Definition Language corresponden a aquellas
que intervienen de alguna manera el esquema de la BBDD o manipulan
nuevos objetos adyacentes a las tablas. Las declaraciones DDL suelen
empezar con un comando \texttt{CREATE} en caso de que se crean objetos,
\texttt{ALTER} para modificarlos y \texttt{DROP} para borrarlos.

Por ejemplo:

\begin{verbatim}
CREATE TABLE mitabla (campo1 tipo1, campo2 tipo2 ... )
DROP TABLE mitabla 
\end{verbatim}

ALTER TABLE cambia la estructura de una tabla. Por ejemplo, puede
agregar o eliminar columnas, crear o borrar índices, cambiar el tipo de
columnas existentes o cambiar el nombre de las columnas o de la tabla en
sí.

Ejemplo de agregar nueva columna:

\begin{verbatim}
    ALTER TABLE mitabla
    ADD camponuevo tipo
\end{verbatim}

También puede ir acompañado de borrar columnas:

\begin{verbatim}
ALTER TABLE mitabla DROP COLUMN nombre_columna1, DROP COLUMN nombre_columna2...
\end{verbatim}

Para borrar una tabla, usar comando DROP TABLE. Para vaciar usar
TRUNCATE TABLE:

\begin{verbatim}
    DROP TABLE nombre-tabla 
    TRUNCATE TABLE nombre-tabla
\end{verbatim}

Ojo: operaciones CREATE/ALTER/DROP requieren permisos de escritura
\end{frame}

\begin{frame}[fragile]{Inserción de registros y populating}
\protect\hypertarget{inserciuxf3n-de-registros-y-populating}{}
\texttt{INSERT} crea una fila con cada columna establecida en su valor
predeterminado

\texttt{INSERT\ INTO\ tbl\_name\ (\ campos...)\ VALUES(\ valores...);}

Es decir, se cargan una lista de campos determinados (que pueden ser
todos) y sus valores respectivos. Un valor de un campo también se le
conoce como \emph{Instancia}.

\begin{itemize}
\tightlist
\item
  Las cadenas de texto siempre deben ir entre comillas.
\item
  Por ejemplo las fechas en SQL son en formato yyyy-mm-dd. Deben ir
  entre comillas como cadenas de texto.
\item
  Los números decimales separan el número con la parte decimal con un
  punto.
\end{itemize}
\end{frame}

\begin{frame}[fragile]{Sintaxis básica de SQL}
\protect\hypertarget{sintaxis-buxe1sica-de-sql}{}
SQL utiliza una sintaxis similar a otros lenguajes de programación,
compuesta por palabras clave, operadores y expresiones. Las sentencias
SQL suelen estar compuestas por cláusulas como SELECT, FROM, WHERE,
ORDER BY y GROUP BY, y se terminan con un punto y coma (;).

\begin{verbatim}
    --Sintaxis consulta
    SELECT campo1, campo2, ...
    FROM tabla
    WHERE condicion
    ORDER BY campo1 ASC/DESC
    --Seleccionar toda una tabla
    SELECT * FROM tabla
\end{verbatim}
\end{frame}

\begin{frame}[fragile]{Sentencia SELECT}
\protect\hypertarget{sentencia-select}{}
La sentencia SELECT se utiliza para recuperar datos de una o más tablas.
Suele estar compuesta por la palabra clave SELECT, seguida de una lista
de columnas a recuperar, y la cláusula FROM, que especifica la tabla o
tablas de las que recuperar los datos.

Ejemplo:

\texttt{SELECT\ columna1,\ columna2,\ columna3\ FROM\ tabla1;}

Esta sentencia recupera los valores de columna1, columna2 y columna3 de
tabla1.
\end{frame}

\begin{frame}[fragile]{Filtrado de datos con la cláusula WHERE}
\protect\hypertarget{filtrado-de-datos-con-la-cluxe1usula-where}{}
La cláusula WHERE se utiliza para filtrar datos en función de ciertas
condiciones. Suele utilizarse en conjunción con la sentencia SELECT y
permite recuperar solo las filas que coinciden con una condición
específica.

\texttt{SELECT\ columna1,\ columna2\ FROM\ tabla1\ WHERE\ columna3\ =\ \textquotesingle{}valor\textquotesingle{}}

Esta sentencia recupera los valores de columna1 y columna2 de tabla1
donde columna3 es igual a `valor'.

La cláusula \texttt{WHERE} involucra condiciones. En general una
condicion es una proposición lógica, es decir un enunciado cuyo valor es
verdadero o falso. Siendo \texttt{WHERE} una instrucción en bloque (como
\texttt{SELECT} también), evaluará qué registros de un campo cumplen con
la condición, y los filtrará del bloque final.

\begin{verbatim}
    --CONDICIONALES
    --operadores
    WHERE campo > valor -- mayor que
    WHERE campo < valor --menor que
    WHERE campo >= valor --mayor o igual
    WHERE campo <= valor --menor o igual
    WHERE campo <> valor -- distinto
    WHERE campo = valor --igual
    WHERE campo LIKE patron (veremos en Seccion 2)
    WHERE campo IN (valor1, valor2...)--si el campo esta en un conjunto de valores
    WHERE campo BETWEEN rangomin AND rangomax --si el campo esta entre rangomin y rango max
    WHERE campo IS NULL -- filas en donde el campo seleccionado es null
    WHERE campo IS NOT NULL --filas donde el campo no es null (vacio)
\end{verbatim}

Respecto a lo anterior, entenderemos los valores \texttt{NULL}, como un
dato vacío que posee algún campo para un registro específico en un
tabla. Es decir, una celda donde no hay nada (ni siquiera espacios en
blanco, por lo que una celda en blanco podría no ser \texttt{NULL}; en
general las celdas vacías mostrarán un \texttt{NULL} para evitar
confusiones).

Como \texttt{WHERE} evalúa proposiciones lógicas, sus valores de verdad
pueden ser sujeto a los operadores tradicionales lógicos (de conjunción,
disyunción, negación).

\begin{verbatim}
    -Operadores Logicos
    --operador AND
    SELECT campo1, campo2, ...
    FROM tabla
    WHERE condicion1 AND condicion2 AND condicion3...
    --operador OR
    SELECT campo1, campo2, ...
    FROM tabla
    WHERE condicion1 OR condicion2 OR condicion3...
    --operador NOT
    SELECT campo1, campo2, ...
    FROM tabla
    WHERE NOT condicion;
\end{verbatim}
\end{frame}

\begin{frame}[fragile]{Ordenación de datos con la cláusula ORDER BY}
\protect\hypertarget{ordenaciuxf3n-de-datos-con-la-cluxe1usula-order-by}{}
La cláusula ORDER BY se utiliza para ordenar los resultados de una
consulta en orden ascendente o descendente. Suele utilizarse en
conjunción con la sentencia SELECT y permite especificar la columna o
columnas por las que ordenar.

\texttt{SELECT\ columna1,\ columna2\ FROM\ tabla1\ ORDER\ BY\ columna1\ ASC;}

Esta sentencia recupera los valores de columna1 y columna2 de tabla1
ordenados por columna1 en orden ascendente.
\end{frame}

\begin{frame}[fragile]{Comando DISTINCT}
\protect\hypertarget{comando-distinct}{}
Dentro de una tabla, una columna a menudo contiene muchos valores
duplicados; a veces solo desea enumerar los valores distintos. El
comando DISTINCT elimina repeticiones de una tupla de datos en una fila.

\begin{verbatim}
SELECT DISTINCT columna1, columna2, ...
FROM tabla_nombre;
\end{verbatim}
\end{frame}
