\hypertarget{clase-6-tuxf3picos-avanzados-de-sql-e-introducciuxf3n-a-bases-nosql}{%
\section{Clase 6 : Tópicos avanzados de SQL e Introducción a bases
NoSQL}\label{clase-6-tuxf3picos-avanzados-de-sql-e-introducciuxf3n-a-bases-nosql}}

\begin{frame}[fragile]{Pivot Tables}
\protect\hypertarget{pivot-tables}{}
Como sabemos, el instrumento de excelencia para operar con tablas
dinámicas es MsExcel. Sin embargo, cuando los volumenes de datos superan
el orden de 10 millones de registros, una tabla dinámica en excel
comenzará a ponerse lenta e inmanejable. Para esto existe la función
PIVOT que emula las funciones de excel mencionadas:

\begin{verbatim}
SELECT  <columna_no_pivote>,
        <lista_columnas_a_pivotear>
FROM
(<SELECT query que produce datos>)
AS <nombre_alias>
PIVOT
(
<funcion de agregado>(<columna agregado>)
FOR
[<campo desde el cual sale lista_columnas_a_pivotear>]
IN ( [ <lista_columnas_a_pivotear> ] )
) AS <alias_pivot>
\end{verbatim}

Por ejemplo:

\begin{verbatim}
-- Creating a pivot table
SELECT category, [2019], [2020], [2021]
FROM mytable
PIVOT (SUM(sales) FOR year IN ([2019], [2020], [2021])) AS pvt
\end{verbatim}
\end{frame}

\begin{frame}[fragile]{Introducción a los formatos de data XML y JSON}
\protect\hypertarget{introducciuxf3n-a-los-formatos-de-data-xml-y-json}{}
JSON y XML son dos formatos de datos utilizados para intercambiar
información entre diferentes aplicaciones y sistemas.

JSON (JavaScript Object Notation) es un formato de texto ligero que se
utiliza comúnmente para enviar y recibir datos en aplicaciones web. Es
fácil de leer y escribir, y se utiliza ampliamente en aplicaciones web
modernas. Los datos se almacenan en pares clave-valor y se pueden anidar
para representar estructuras más complejas.

XML (Extensible Markup Language) es un lenguaje de marcado que se
utiliza para almacenar y transportar datos. Es un formato más antiguo
que JSON y se utiliza en una amplia variedad de aplicaciones. Los datos
se almacenan en etiquetas que describen la estructura y el contenido de
los datos. XML también permite definir etiquetas personalizadas y crear
documentos bien formados que pueden ser validados para asegurarse de que
cumplen con un conjunto de reglas específicas.

En resumen, JSON y XML son formatos de datos utilizados para
intercambiar información entre diferentes aplicaciones y sistemas. JSON
es un formato más ligero y fácil de leer y escribir, mientras que XML es
un lenguaje de marcado más antiguo que permite una mayor flexibilidad en
la definición de etiquetas personalizadas y la validación de documentos.

Ejemplo de objeto JSON:

\begin{verbatim}
{
   "firstName": "John",
   "lastName": "Doe",
   "age": 30,
   "address": {
       "street": "123 Main St",
       "city": "Anytown",
       "state": "CA",
       "zip": "12345"
   },
   "phoneNumbers": [
       {
           "type": "home",
           "number": "555-1234"
       },
       {
           "type": "work",
           "number": "555-5678"
       }
   ]
}
\end{verbatim}

Este objeto JSON representa información de una persona, incluyendo su
nombre, edad, dirección y números de teléfono. Los valores se almacenan
en pares clave-valor, y algunos valores, como la dirección y los números
de teléfono, se almacenan como objetos anidados.

Ejemplo de documento XML:

\begin{verbatim}
<person>
   <firstName>John</firstName>
   <lastName>Doe</lastName>
   <age>30</age>
   <address>
      <street>123 Main St</street>
      <city>Anytown</city>
      <state>CA</state>
      <zip>12345</zip>
   </address>
   <phoneNumbers>
      <phoneNumber type="home">555-1234</phoneNumber>
      <phoneNumber type="work">555-5678</phoneNumber>
   </phoneNumbers>
</person>
\end{verbatim}

Este documento XML también representa información de una persona, con
valores almacenados en etiquetas que describen su contenido. Los valores
anidados se representan como etiquetas anidadas, y los atributos, como
el tipo de número de teléfono, se almacenan como atributos en las
etiquetas correspondientes.

Es importante tener en cuenta que los objetos JSON y los documentos XML
pueden variar en complejidad y estructura, y que estas son solo
representaciones simples.
\end{frame}

\begin{frame}[fragile]{Trabajando en SQL con data JSON y XML}
\protect\hypertarget{trabajando-en-sql-con-data-json-y-xml}{}
\begin{verbatim}
-- Working with JSON data
DECLARE @json NVARCHAR(MAX) = '{"name": "John", "age": 30}'
SELECT JSON_VALUE(@json, '$.name') AS name, JSON_VALUE(@json, '$.age') AS age

-- Working with XML data
DECLARE @xml XML = '<person><name>John</name><age>30</age></person>'
SELECT @xml.value('(/person/name)[1]', 'nvarchar(100)') AS name, @xml.value('(/person/age)[1]', 'int') AS age

\end{verbatim}
\end{frame}

\begin{frame}{}
\protect\hypertarget{section}{}
\end{frame}
