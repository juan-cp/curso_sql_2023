\hypertarget{clase-2-sql-nivel-medio}{%
\section{Clase 2: SQL nivel medio}\label{clase-2-sql-nivel-medio}}

\begin{frame}[fragile]{Alias de columna y tabla}
\protect\hypertarget{alias-de-columna-y-tabla}{}
Los alias de columna y tabla se utilizan para cambiar el nombre de las
columnas o tablas en una consulta. Los alias se definen utilizando la
palabra clave AS o simplemente un espacio en blanco.

Ejemplo:

\begin{verbatim}
SELECT column1 AS 'Nombre de columna 1', column2 AS 'Nombre de columna 2' FROM table1 AS t1;
\end{verbatim}
\end{frame}

\begin{frame}[fragile]{Uso de funciones agregadas: SUM, AVG, MAX, MIN,
COUNT}
\protect\hypertarget{uso-de-funciones-agregadas-sum-avg-max-min-count}{}
Las funciones agregadas se utilizan para realizar cálculos en conjuntos
de datos. Incluyen funciones como SUM, AVG, MAX, MIN y COUNT. Estas
funciones se pueden utilizar para calcular totales, promedios, valores
máximos y mínimos y recuentos de filas.

Ejemplo:

\begin{verbatim}
SELECT SUM(columna1) FROM tabla1;
\end{verbatim}

Esta instrucción calcula la suma de la columna 1 de la tabla 1.
\end{frame}

\begin{frame}[fragile]{Agrupar resultados con la cláusula GROUP BY y uso
de HAVING}
\protect\hypertarget{agrupar-resultados-con-la-cluxe1usula-group-by-y-uso-de-having}{}
La cláusula \texttt{GROUP\ BY} se utiliza para agrupar filas que tienen
el mismo valor en una o más columnas. La cláusula \texttt{HAVING} se
utiliza para filtrar grupos basados en una condición.

Ejemplo:

\begin{verbatim}
SELECT column1, COUNT(column2) 
FROM table1 
GROUP BY column1 
HAVING COUNT(column2) > 1;
\end{verbatim}

Esta consulta agrupa los valores de column1 en table1 y cuenta el número
de valores de column2 para cada grupo. Luego, la cláusula
\texttt{HAVING} filtra los grupos que tienen más de un valor de column2.
\end{frame}

\begin{frame}[fragile]{Subconsultas y consultas anidadas}
\protect\hypertarget{subconsultas-y-consultas-anidadas}{}
Las subconsultas se utilizan para realizar consultas dentro de una
consulta. Las subconsultas pueden ser utilizadas en cláusulas WHERE,
SELECT, FROM y HAVING.

Ejemplo:

\begin{verbatim}
SELECT column1, column2 
FROM table1 
WHERE column1 IN (SELECT column1 FROM table2 WHERE column2 = 'value');
\end{verbatim}

Esta consulta devuelve los valores de column1 y column2 de table1 donde
column1 está presente en la subconsulta, que selecciona los valores de
column1 de table2 donde column2 es igual a `value'.
\end{frame}

\begin{frame}[fragile]{Expresiones CASE}
\protect\hypertarget{expresiones-case}{}
Las expresiones CASE se utilizan para realizar evaluaciones
condicionales en una consulta.

Ejemplo:

\begin{verbatim}
SELECT column1, column2, column3 
CASE 
  WHEN column3 > 0 THEN 'Positive' 
  WHEN column3 < 0 THEN 'Negative'
  ELSE 'Zero' 
END AS 'Column4' 
FROM table1;
\end{verbatim}

Esta consulta selecciona los valores de column1, column2 y una nueva
columna Column4 que se calcula utilizando una expresión CASE que evalúa
los valores de column3.
\end{frame}

\begin{frame}[fragile]{Expresiones de tabla comunes (CTE)}
\protect\hypertarget{expresiones-de-tabla-comunes-cte}{}
Las expresiones de tabla comunes se utilizan para definir subconsultas
que pueden ser referenciadas varias veces en un script.

Ejemplo:

\begin{verbatim}
WITH cte AS (SELECT column1, column2 FROM table1 WHERE column3 > 0)
SELECT column1, COUNT(column2) 
FROM cte 
GROUP BY column1;
\end{verbatim}

Esta consulta define una expresión de tabla común (cte) y realiza un
conteo sobre ella.
\end{frame}
